\documentclass{beamer}
\usepackage{algorithm}
\usepackage{algorithmic}
\usepackage{algpseudocode}
\usepackage{amsmath}
\usetheme{default}
\begin{document}

%Information to be included in the title page:
\title{Fast Label Embeddings via Randomized Linear Algebra}

\author{Author: Paul Mineiro and Nikos Karampatziakis \\ Presented By: Sriteja Reddy Pashya and Romica Raisinghani}

\institute{MA8.401 Topics in Applied Optimization}

\date{$5^\text{th}$ December 2023}
\frame{\titlepage}


\begin{frame}
\frametitle{Motivation and Problem Statement}
\begin{itemize}
    \item Modern multiclass and multilabel problems are characterized by increasingly large output spaces.
    \item The goal is to improve both computational and statistical efficiency in dealing with these problems.
    \item The authors propose the use of label embeddings to tackle problems with large output spaces.
    \item A fast label embedding algorithm is introduced that works in both the multiclass and multilabel settings.
    \item The algorithm uses techniques from randomized linear algebra to develop an efficient and scalable method for constructing the embeddings.
    \item The techniques are demonstrated on two large-scale public datasets, where they obtain state-of-the-art results.
\end{itemize}
\end{frame}


\begin{frame}
\frametitle{Algorithm Notations}
Let us have a look at some of the notations that will be used in the algorithm:
\begin{itemize}
    \item Vectors are denoted by lowercase letters $ x, y, . . .$.
    \item Matrices are denoted by uppercase letters $ W, W, . . .$.
    \item Input dimension: $d$
    \item Output dimension: $c$
    \item Embedding dimension: $k$
    \item $ X \in \mathbb{R}^{m \times n}$ denotes an $m \times n$ matrix.
    \item \(\|X\|_F\) represents the Frobenius norm of matrix $X$.
\end{itemize}
For a given matrix $A$ with dimensions $m \times n$, Frobenius norm is computed as follows:
\begin{equation*}
    \|A\|_F = \sqrt{\sum_{i=1}^m \sum_{j=1}^n |a_{ij}|^2}
\end{equation*}
\end{frame}

\begin{frame}{Algorithm Notations Continued}
\begin{itemize}
    \item For multiclass problems, $y$ is a one-hot vector.
    \item For multilabel problems, $y$ is a binary vector.
    \item $X^\dagger$ denotes the pseudoinverse of matrix $X$.
    \item $\Pi_{X,L}$ represents the projection onto the left singular subspace of $X$.
    \item $X_{1:k}$ denotes the matrix obtained by taking the first $k$ columns of $X$.
    \item $X^*$ denotes a matrix obtained by solving an optimization problem over the matrix parameter $X$.
    \item The expectation of a random variable $v$ is denoted by $\mathbb{E}[v]$.
\end{itemize}
\end{frame}

\begin{frame}
\frametitle{Regular PCA Algorithm}
\begin{algorithm}[H]
\caption{Regular PCA}
\Function{PCA}{$k, X \in \mathbb{R}^{n \times d}$}
    \State $C \gets X^TX$ \Comment{Compute the covariance matrix}
    \State $(U, \Sigma, V^T) \gets \text{SVD}(C)$ \Comment{Singular Value Decomposition}
    \State $U_k \gets U[:, 1:k]$ \Comment{Select the first k columns of U}
    \State $\Sigma_k \gets \Sigma[1:k, 1:k]$ \Comment{Select the top k x k block of $\Sigma$}
    \State $V_k^T \gets V^T[1:k, :]$ \Comment{Select the first k rows of $V^T$}
    \State $Y \gets X V_k$ \Comment{Project X onto the space spanned by the top k eigenvectors}
    \State \Return $(Y, \Sigma_k)$
\EndFunction
\end{algorithm}
\end{frame}


\begin{frame}
\frametitle{Randomized PCA Algorithm}
\begin{algorithm}[H]
\caption{Randomized PCA}
\Function{RPCA}{$k, X \in \mathbb{R}^{n \times d}$}
    \State $(p, q) \gets (20, 1)$ \Comment{Hyperparameters}
    \State $Q \gets \text{randn}(d, k + p)$
    \For{$i \in \{1, \ldots, q\}$}
        \State $\Psi \gets X^TXQ$ \Comment{$\Psi$ computed in one pass}
        \State $Q \gets \text{orthogonalize}(\Psi)$
    \EndFor
    \State $F \gets (X^TXQ)^T(X^TXQ)$
    \State $(V, \Sigma^2) \gets \text{eig}(F, k)$
    \State $V \gets (X^TXQ)V\Sigma^\dagger$
    \State \Return $(V, \Sigma)$
\EndFunction
\end{algorithm}
\end{frame}



\begin{frame}
\frametitle{Explanation: RPCA Algorithm}
\begin{itemize}
    \item Algorithm starts with a function RPCA that takes two parameters i.e $k$(number of desired principal components) and input matrix $X$(containing $n$ examples and $d$ features).
    \item Hyper-parameters $p$ and $q$ rarely need adjustment. $p$ is used to oversample the number of random vectors to ensure a good range approximation, and $q$ is the number of times the range finding loop will run.
    \item Generate a matrix $Q$ of dimensions $d \times (k + p)$ filled with random numbers. This matrix is used to probe the range of the matrix $X^T X$.
    \item Algorithm runs a loop $q$ times. In each iteration:\\
    1. Compute the matrix $\psi$ as the product of $X^T X$ and $Q$.\\
    2. Update $Q$ to an orthogonal basis for the range of $\psi$.

\end{itemize}
\end{frame}

\begin{frame}{Explanation: RPCA Algorithm (Continued)}
    \begin{itemize}
        \item After completing the $q$ iterations, compute the matrix $F$ as the product of $(X^T XQ)^T$ and $(X^T XQ)$. The matrix $F$ is of size $(k + p) \times (k + p)$ and is small relative to the original data matrix.
        \item Perform eigendecomposition on the small matrix $F$ to obtain the top $k$ eigenvectors and eigenvalues. The eigenvectors are stored in $V$, and the eigenvalues (squared) are stored in $\sum^2$.
        \item  Multiply $(X^T XQ)$ with $V$ and scale by the pseudoinverse of $\sum$ (denoted $\sum^\dagger$). This step backs out the solution to find the approximate principal components.
        \item Return the matrices $V$ (eigenvectors) and $\sum$ (square root of the eigenvalues), which represent the principal components and their associated variances.
    \end{itemize}
\end{frame}


\begin{frame}
\frametitle{Rembrandt Algorithm}
\begin{algorithm}[H]
\caption{: \textbf{R}esponse \textbf{EMB}edding via \textbf{RAND}omized \textbf{T}echniques}
\Function{Rembrandt}{$k, X \in \mathbb{R}^{n \times d}, Y \in \mathbb{R}^{n \times c}$}
    \State $(p, q) \gets (20, 1)$ \Comment{Hyperparameters}
    \State $Q \gets \text{randn}(c, k + p)$
    \For{$i \in \{1, \ldots, q\}$}
        \State $Z \gets \text{argmin} \Vert YQ - XZ \Vert_F^2$
        \State $Q \gets \text{orthogonalize}(YXZ)$
    \EndFor
    \State $F \gets (YXZ)(YXZ)^T$
    \State $(V, \Sigma^2) \gets \text{eig}(F, k)$
    \State $V \gets (YXZ)V\Sigma^\dagger$
    \State \Return $(V, \Sigma)$
\EndFunction
\end{algorithm}
\end{frame}

% \begin{frame}
% \frametitle{Explanation: Rembrandt Algorithm}

% \begin{block}{Optimal Squared Loss Prediction}
% We aim to find an optimal squared loss predictor for a high-cardinality target vector \( y \in \mathbb{R}^c \), which is linear in a high dimensional feature vector \( x \in \mathbb{R}^d \).
% \end{block}

% \begin{block}{Constraints and Challenges}
% Due to concerns of sample complexity, a low-rank constraint is imposed on the weight matrix, leading to the problem:
% \[ W^* = \underset{W \in \mathbb{R}^{d \times c} | \text{rank}(W) \leq k}{\arg\min} \| Y - XW \|_F^2 \]
% where \( Y \in \mathbb{R}^{n \times c} \) and \( X \in \mathbb{R}^{n \times d} \) represent the target and design matrices respectively.
% \end{block}

% % \begin{block}{Generalization}
% % This issue is a specific instance of a broader problem in the field, focusing on balancing computational efficiency and statistical accuracy in extensive output spaces.
% % \end{block}
% \end{frame}

% \begin{frame}
% \frametitle{Explanation: Rembrandt Algorithm}
% The Rembrandt algorithm computes an optimal weight matrix \( W^* \) using a three-step procedure:

% \begin{enumerate}
%     \item \textbf{Projection}: It projects the target matrix \( Y \) onto \( k \) dimensions using the top right singular vectors of \( \Pi_{X,L} Y \), where \( \Pi_{X,L} \) is the projection onto the left singular subspace of \( X \).
%     \item \textbf{Least Squares Fit}: Then it fits the projected labels to the features using a least squares fit, which is a standard approach for finding the best-fit linear model that minimizes the sum of the squares of the errors.
%     \item \textbf{Mapping Predictions}: Finally, it maps these predictions back to the original output space using the transpose of the top \( k \) right singular vectors of \( \Pi_{X,L} Y \).
% \end{enumerate}

% This process efficiently approximates \( W^* =  \) without having to compute it directly, thus saving on computational resources while still capturing the essential structure of the data.
% \end{frame}

\begin{frame}
\frametitle{Optimal Squared Loss Predictor with Low-Rank Constraint}
\begin{block}{Problem Formulation}
We aim to find a weight matrix \( W^* \) for predicting a high-cardinality target vector \( y \in \mathbb{R}^c \) from a high-dimensional feature vector \( x \in \mathbb{R}^d \). The challenge is to minimize the squared loss, \( \lVert Y - XW \rVert_F^2 \), under a low-rank constraint \( \text{rank}(W) \leq k \), where \( Y \) and \( X \) are the target and design matrices, respectively.
\end{block}

\begin{block}{Solution via SVD}
The solution \( W^* \) is obtained through the projection \( \Pi_{X,L} \) onto the left singular subspace of \( X \), and involves the optimal Frobenius norm rank-\( k \) approximation. The expression for \( W^* \) is derived using SVD, simplifying to \( W^* = X^\dagger(Y V_{1:k})V_{1:k}^T \).
\end{block}

\begin{block}{Reference}
Friedland, S., Torokhti, A.: Generalized rank-constrained matrix approximations. SIAM Journal on Matrix Analysis and Applications 29(2), 656–659 (2007).
\end{block}
\end{frame}

% \begin{frame}
% \frametitle{Rembrandt Algorithm: Computation Steps}
% \begin{algorithm}[H]
% \caption{Computation Steps for \( W^* \)}
% \begin{enumerate}
%     \item Project \( Y \) onto \( k \) dimensions using the right singular vectors of \( \Pi_{X,L}Y \).
%     \item Perform a least squares fit with the projected labels and features in \( X \).
%     \item Map the predictions back to the original output space with the right singular vectors.
% \end{enumerate}
% \end{algorithm}
% \end{frame}

\begin{frame}
\frametitle{Rembrandt Algorithm: Computation Steps}
The Rembrandt algorithm computes an optimal weight matrix \( W^* \) using a three-step procedure:

\begin{enumerate}
    \item \textbf{Projection}: It projects the target matrix \( Y \) onto \( k \) dimensions using the top right singular vectors of \( \Pi_{X,L} Y \), where \( \Pi_{X,L} \) is the projection onto the left singular subspace of \( X \).
    \item \textbf{Least Squares Fit}: Then it fits the projected labels to the features using a least squares fit, which is a standard approach for finding the best-fit linear model that minimizes the sum of the squares of the errors.
    \item \textbf{Mapping Predictions}: Finally, it maps these predictions back to the original output space using the transpose of the top \( k \) right singular vectors of \( \Pi_{X,L} Y \).
\end{enumerate}

This process efficiently approximates \( W^* =  \) without having to compute it directly, thus saving on computational resources while still capturing the essential structure of the data.
\end{frame}

% \begin{frame}
% \frametitle{Understanding Label Embeddings via SVD}
% \begin{block}{Optimal Unconstrained Model}
% An optimal model \( Z^* \) is found by minimizing the squared Frobenius norm of \( Y - XZ \). The term \( \Pi_{X,L}Y \) represents the predictions of this model, denoted \( \hat{Y} \).
% \end{block}

% \begin{block}{Right Singular Vectors as Embeddings}
% The right singular vectors \( V \) of \( \Pi_{X,L}Y \) are the eigenvectors of \( \hat{Y}^T\hat{Y} \), capturing the essential patterns in the model's predictions.
% \end{block}

% \begin{block}{Efficiency in Computation}
% Direct computation of \( Z^* \) is costly. The randomized algorithm sidesteps this by multiplying \( \Pi_{X,L}Y \) with a matrix \( Q \), optimizing the process and achieving the same result.
% \end{block}

% \begin{block}{Mathematical Formulation}
% The product \( \Pi_{X,L}YQ \) is computed as \( XZ^*Q \), where \( Z^*Q \) minimizes the squared Frobenius norm of \( YQ - XZ \).
% \end{block}

% \begin{block}{Footnote}
% If \( X = U_{X}\Sigma_{X}V_{X}^T \) is the SVD of \( X \), then \( \Pi_{X,L} = U_{X}U_{X}^T \).
% \end{block}
% \end{frame}

\begin{frame}
\frametitle{Label Embedding and Randomized Algorithms}
The right singular vectors of \( \Pi_{X,L}Y \) are utilized for label embedding, motivated by the predictions of the optimal unconstrained model:

\[ Z^* = \arg\min_{Z \in \mathbb{R}^{d \times c}} \|Y - XZ\|_F^2, \]
\[ \Pi_{X,L}Y = XZ^* \equiv \hat{Y}. \]

These vectors, \( V \), of \( \Pi_{X,L}Y \) are the eigenvectors of \( \hat{Y}^T\hat{Y} \), the matrix of outer products of the model's predictions. Avoiding the direct computation of \( Z^* \), the algorithm finds \( \Pi_{X,L}YQ = XZ^*Q \) by solving:

\[ Z^*Q = \arg\min_{Z \in \mathbb{R}^{d \times k}} \|YQ - XZ\|_F^2. \]
\end{frame}


\begin{frame}
\frametitle{Bias-Variance Tradeoff and Model Generalization}
\begin{block}{Squared Loss Minimization}
Squared loss, as a proper scoring rule, is minimized at the conditional mean. With sufficient data and model flexibility, \( \frac{1}{n} \hat{Y}^T\hat{Y} \) converges to \( \mathbb{E}[\mathbb{E}[y|x]^T \mathbb{E}[y|x]] \).
\end{block}
\begin{block}{Law of Large Numbers}
This convergence is assured by the strong law of large numbers.
\end{block}
\begin{equation}
    \frac{1}{n} \hat{Y}^T\hat{Y} \xrightarrow{a.s.} \mathbb{E}[\mathbb{E}[y|x]^T \mathbb{E}[y|x]]
\end{equation}
\end{frame}


\begin{frame}
\frametitle{Eigendecomposition and Embedding Insights}
\begin{block}{Eigendecomposition}
An embedding based on the eigendecomposition of \( \mathbb{E}[\mathbb{E}[y|x]^T \mathbb{E}[y|x]] \) is not actionable but insightful, approximating it by the empirical label covariance \( Y^TY \).
\end{block}
\begin{block}{Generalization in High-Dimensional Spaces}
For multiclass or multilabel cases, the low-rank constraint might not ensure good generalization if the model is overly flexible. The eigendecomposition may only capture the most frequent labels, missing out on label co-occurrence patterns.
\end{block}
\begin{block}{Model Tuning}
To better approximate \( \mathbb{E}[Y|X] \) over the observed \( Y \), one must trade off variance for bias, tuning the bias-variance tradeoff by the choice of model features, as used in Algorithm 2.
\end{block}
\end{frame}


% \begin{frame}
% \frametitle{Results: LSHTC and ODP Datasets}
% \begin{itemize}
%     \item \textbf{LSHTC Dataset:} 
%     \begin{itemize}
%         \item Rembrandt with Independent Logistic Regression (RE + ILR) achieved a precision-at-1 of $53.39\%$ for \( k = 800 \).
%         \item This method outperforms FastXML and LPSR-NB, showing the efficiency of label embeddings$&\text{\#}8203;``[oaicite:1]``&\text{\#}8203;$.
%     \end{itemize}
%     \item \textbf{ODP Dataset:} 
%     \begin{itemize}
%         \item Rembrandt and logistic regression (RE + LR) resulted in the lowest test error of 83.15% for \( k = 300 \).
%         \item This result is to the best of the authors' knowledge, the best published result on this dataset $& \text{\#}8203;``[oaicite:0]``&\text{\#}8203;$.
%     \end{itemize}
% \end{itemize}
% \end{frame}

% \begin{frame}
% \frametitle{Results: ALOI Dataset}
% \begin{itemize}
%     \item \textbf{ALOI Dataset:}
%     \begin{itemize}
%         \item Performance of Rembrandt + Logistic Regression (RE + LR) and PCA + Logistic Regression (PCA + LR) showed a test error of $9.7\%$.
%         \item The study shows that conventional methods like one-against-all (OAA) and logistic regression (LR) are infeasible on single machines for larger datasets.
%     \end{itemize}
% \end{itemize}
% \end{frame}


\begin{frame}
\frametitle{Conclusion}
\begin{itemize}
    \item The proposed label embedding techniques significantly improve computational and statistical efficiency in large output space problems.
    \item Experiments on large-scale datasets demonstrate that the Rembrandt algorithm coupled with logistic regression outperforms other methods.
    \item The results highlight the potential of label embeddings in dealing with multiclass and multilabel problems, achieving state-of-the-art performance.
\end{itemize}
\end{frame}


\begin{frame}
\begin{align*}
    \text{THANK YOU !}
\end{align*}
\end{frame}
\end{document}